% Search for all the places that say "PUT SOMETHING HERE".

\documentclass[11pt]{article}
\usepackage{amsmath,textcomp,amssymb,geometry,graphicx,enumerate}

\def\Name{Kevin Chau}  % Your name
\def\SID{23816929}  % Your student ID number
\def\Login{cs170-pz} % Your login (your class account, cs170-xy)
\def\Homework{1} % Number of Homework
\def\Session{Spring 2015}


\title{CS170--Spring 2015 --- Solutions to Homework \Homework}
\author{\Name, SID \SID, \texttt{\Login}}
\markboth{CS170--\Session\  Homework \Homework\ \Name}{CS170--\Session\ Homework \Homework\ \Name, \texttt{\Login}}
\pagestyle{myheadings}
\date{}

\newenvironment{qparts}{\begin{enumerate}[{(}a{)}]}{\end{enumerate}}
\def\endproofmark{$\Box$}
\newenvironment{proof}{\par{\bf Proof}:}{\endproofmark\smallskip}

\textheight=9in
\textwidth=6.5in
\topmargin=-.75in
\oddsidemargin=0.25in
\evensidemargin=0.25in


\begin{document}
\maketitle

Collaborators: Howard Chiao, Matthew Magsombol, GangHoon Kim

\section*{1. Getting started}
\item
I, Kevin Chau, understand the course policies.



\newpage
\section*{2. Compare growth rates}
\begin{qparts}
\item
f(n)=\omega(g(n))

\item
YOUR ANSWER GOES HERE

\item
YOUR ANSWER GOES HERE

\item
YOUR ANSWER GOES HERE

\item
YOUR ANSWER GOES HERE

\item
YOUR ANSWER GOES HERE

\item
YOUR ANSWER GOES HERE

\item
YOUR ANSWER GOES HERE

\item
YOUR ANSWER GOES HERE

\item
YOUR ANSWER GOES HERE
\end{qparts}


\newpage
\section*{3.}
YOUR ANSWER GOES HERE


\newpage
\section*{4.}
YOUR ANSWER GOES HERE


\newpage
\section*{5.}
\begin{qparts}
\item
From here we must show that n\^n = O(n!\^c). We can do this by induction. To show that the base case works, we notice that c\geq2, otherwise there is no n which could satisfy n\^n < n!. The smallest numbers c and n that we can choose to satisfy the base case turn out to be c=2 and n=3 (3\^3=27 < 3!\^2=36). Next we make the inductive hypothesis n\^n < n!\^c. Now for the induction: (n+1)\^(n+1)= (n+1)(n+1)\^n < (n+1)(n\^(n+1))= (n+1)*n*n\^n < (n+1)*n*(n!\^c) , where we used the hypothesis in the last inequality. Next notice that (n+1)*n*(n!\^c) < (n+1)(n+1)(n!\^c) \leq ((n+1)\^c)*(n!\^c), where again we must have c\geq2 for the last inequality to hold. But ((n+1)\^c)*(n!\^c) = (n+1)!\^c, thus we have shown (n+1)\^(n+1) < (n+1)!\^c which proves that the claim holds for all n.

\end{qparts}

\newpage
\section*{6.}
\begin{qparts}
\item
YOUR ANSWER GOES HERE

\item
YOUR ANSWER GOES HERE

\item
YOUR ANSWER GOES HERE
\end{qparts}


\newpage
\section*{7.}
YOUR ANSWER GOES HERE


\end{document}
